\documentclass{article}
\usepackage{graphicx} %
\usepackage{tikz}
\usepackage{amsmath}
\usepackage{amssymb}
\usetikzlibrary{arrows}

\title{Gilbran Mahdavikia Raja}
\author{5025241134}

\begin{document}

\begin{flushleft}
    Gilbran Mahdavikia Raja \\
    5025241134 \\
\end{flushleft}
{\Large \textbf{5.7}}
\begin{description}
    \item[$47.$] 
    Diberikan \( f(x) = x^{2/3} \), \( a = -1 \), dan \( b = 8 \).
    
    \begin{itemize}
        \item[(a)] Tunjukkan bahwa tidak ada titik \( c \) dalam \( (a, b) \) sedemikian hingga:
        \[
        f'(c) = \frac{f(b) - f(a)}{b-a}
        \]

        \item[$\Leftrightarrow$] $
        f(a) = f(-1) = (-1)^{2/3} = 1, \quad f(b) = f(8) = 8^{2/3} = 4.
        $
        \item[$\Leftrightarrow$]
        $
            \frac{f(b) - f(a)}{b - a} = \frac{4 - 1}{8 - (-1)} = \frac{3}{9} = \frac{1}{3}.
        $
        \item[]$f'(x) = \frac{2}{3} x^{-1/3} = \frac{2}{3 \sqrt[3]{x}}.
        $
        \item[$\Leftrightarrow$]
        $
        f'(c) = \frac{1}{3} \implies \frac{2}{3 \sqrt[3]{c}} = \frac{1}{3}.
        $
        \item[$\Leftrightarrow$]Selesaikan untuk \( c \):
        $
        \frac{2}{\sqrt[3]{c}} = 1 \implies \sqrt[3]{c} = 2 \implies c = 2^3 = 8.
        $

        Namun, \( c = 8 \) berada di ujung interval \([a, b]\), sehingga tidak termasuk dalam interval terbuka \( (a, b) \). Oleh karena itu, tidak ada \( c \) dalam \( (a, b) \) yang memenuhi \( f'(c) = \frac{1}{3} \).


        \item[(b)] Terangkan mengapa hasil bagian (a) tidak melanggar Teorema Nilai Rata-rata.
        \item[] Teorema Nilai Rata-rata menyatakan bahwa jika $ f(x) $ kontinu pada $[a, b]$ dan terdiferensialkan pada $(a, b)$, maka terdapat $ c \in (a, b) $ sedemikian hingga:
        \[
        f'(c) = \frac{f(b) - f(a)}{b - a}.
        \]
        
        Namun, dalam kasus ini:
        \begin{itemize}
            \item Fungsi \( f(x) = x^{2/3} \) kontinu pada \([a, b]\), tetapi tidak terdiferensialkan di \( x = 0 \), karena turunan \( f'(x) \) tidak terdefinisi di \( x = 0 \).
        \end{itemize}
        
        Karena salah satu syarat Teorema Nilai Rata-rata (yaitu terdiferensialkan pada seluruh interval terbuka \((a, b)\)) tidak terpenuhi, maka hasil pada bagian (a) tidak melanggar Teorema Nilai Rata-rata.
        
    \end{itemize}
\end{description}

\end{document}