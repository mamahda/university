\documentclass{article}
\usepackage{graphicx} %
\usepackage{tikz}
\usepackage{amsmath}
\usepackage{enumitem}
\usetikzlibrary{arrows}

\title{Gilbran Mahdavikia Raja}
\author{5025241134}

\begin{document}

\begin{flushleft}
    Gilbran Mahdavikia Raja \\
    5025241134 \\
\end{flushleft}
{\Large \textbf{3.4}}
\begin{description}
    \item[$6.$]
    $h(x) = \frac{2}{x} + \frac{x-2}{x^2-4}$
    \begin{enumerate}[label=(\alph*)]
        \item $h$ mempunyai tiga diskontinuitas yang tidak dapat dihilangkan.\\ salah, karena fungsi $h$ hanya memiliki 2 titik diskontinu yang tidak dapat dihilangkan.
        \item $h$ mempunyai diskontinuitas di $x = 2$ yang dapat dihilangkan. \\ 
        benar, pada \( x = 2 \), fungsi awal tidak terdefinisi. Namun, setelah disederhanakan, fungsi menjadi:
        \[h(x) = \frac{2}{x} + \frac{1}{x+2}\]
        Sehingga diskontinuitas di \( x = 2 \) dapat dihilangkan.
        \item $h$ mempunyai diskontinuitas di $ x = -2 $ yang tidak dapat dihilangkan.\\
        benar, pada $x = -2$ fungsi tidak terdefinisi dan juga tidak dapat dihilangkan.
        \item $h$ diskontinu di $ x = 0 $ dan di $ x = 2 $.\\ benar, karena $h(0)$ dan $h(2)$ tidak ada.
        \item $h$ tidak mempunyai diskontinuitas.\\
        salah, walaupun titik diskontinu di $x = 2$ dapat dihilangkan, namun $x = -2$ dan $x = 0$ tidak dapat dihilangkan.
    \end{enumerate}
\end{description}

\end{document}